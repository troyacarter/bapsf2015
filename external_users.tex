\documentclass[11pt]{article}
\usepackage{lmodern}
\usepackage{amssymb,amsmath}
\usepackage{ifxetex,ifluatex}
\usepackage{fixltx2e} % provides \textsubscript
\ifnum 0\ifxetex 1\fi\ifluatex 1\fi=0 % if pdftex
  \usepackage[T1]{fontenc}
  \usepackage[utf8]{inputenc}
\else % if luatex or xelatex
  \ifxetex
    \usepackage{mathspec}
    \usepackage{xltxtra,xunicode}
  \else
    \usepackage{fontspec}
  \fi
  \defaultfontfeatures{Mapping=tex-text,Scale=MatchLowercase}
  \newcommand{\euro}{€}
\fi
% use upquote if available, for straight quotes in verbatim environments
\IfFileExists{upquote.sty}{\usepackage{upquote}}{}
% use microtype if available
\IfFileExists{microtype.sty}{%
\usepackage{microtype}
\UseMicrotypeSet[protrusion]{basicmath} % disable protrusion for tt fonts
}{}
\ifxetex
  \usepackage[setpagesize=false, % page size defined by xetex
              unicode=false, % unicode breaks when used with xetex
              xetex]{hyperref}
\else
  \usepackage[unicode=true]{hyperref}
\fi
\usepackage[usenames,dvipsnames]{color}
\hypersetup{breaklinks=true,
            bookmarks=true,
            pdfauthor={},
            pdftitle={},
            colorlinks=true,
            citecolor=blue,
            urlcolor=blue,
            linkcolor=magenta,
            pdfborder={0 0 0}}
\urlstyle{same}  % don't use monospace font for urls
\usepackage{longtable,booktabs}
\usepackage{graphicx,grffile}
\makeatletter
\def\maxwidth{\ifdim\Gin@nat@width>\linewidth\linewidth\else\Gin@nat@width\fi}
\def\maxheight{\ifdim\Gin@nat@height>\textheight\textheight\else\Gin@nat@height\fi}
\makeatother
% Scale images if necessary, so that they will not overflow the page
% margins by default, and it is still possible to overwrite the defaults
% using explicit options in \includegraphics[width, height, ...]{}
\setkeys{Gin}{width=\maxwidth,height=\maxheight,keepaspectratio}
\setlength{\parindent}{0pt}
\setlength{\parskip}{6pt plus 2pt minus 1pt}
\setlength{\emergencystretch}{3em}  % prevent overfull lines
\providecommand{\tightlist}{%
  \setlength{\itemsep}{0pt}\setlength{\parskip}{0pt}}
\setcounter{secnumdepth}{0}

\date{}

% Redefines (sub)paragraphs to behave more like sections
\ifx\paragraph\undefined\else
\let\oldparagraph\paragraph
\renewcommand{\paragraph}[1]{\oldparagraph{#1}\mbox{}}
\fi
\ifx\subparagraph\undefined\else
\let\oldsubparagraph\subparagraph
\renewcommand{\subparagraph}[1]{\oldsubparagraph{#1}\mbox{}}
\fi

\setlength{\oddsidemargin}{-0.1in}
\setlength{\topmargin}{-0.52truein} 
\setlength{\textheight}{9.15in} 
\setlength{\textwidth}{6.7in}

\usepackage[T1]{fontenc}
\usepackage{fourier}
\usepackage[sc]{mathpazo}
\linespread{1.05}         % Palatino needs more leading (space between lines)


\usepackage{wrapfig}
\usepackage[square,numbers,sort&compress]{natbib}
\renewcommand{\cite}{\citep}
%\usepackage[psamsfonts]{amssymb}
%\usepackage{palatino}
%\usepackage{mathpazo}

%\usepackage{plasmadefs}

\hyphenation{wave-packet wave-packets}

\title{}

\begin{document}

\section{Current active external user groups}

%\subsection{External users}
%Currently, there are thirteen active external user groups of the facility: four independent experimenter user groups, six theory-driven studies and three topical campaigns.  The four experimental groups and six-theory driven studies are listed below as: ?project name?, ?research team? (leader listed first), ?affiliation?.

\subsection{Independent experimenter user groups}

\begin{enumerate}
\item "Study of Ion Transport in Turbulent Plasmas", W. Heidbrink, R. McWilliams, H. Boehmer (Dept. of Physics, University of California, Irvine.)\\
A moderate energy ( 1 keV.) Lithium ion beam is mounted in the LAPD. The beam spirals along the background magnetic field in an argon or helium plasma. The beam profile will be measured with probes as it moves through localized turbulent layers. The layers are generated with antennas. The beam divergence and energy spread is being studied.

\item "Laser Driven shock waves in the LAPD", C. Niemann, C. Constantin, Phoenix laser group, (Dept. of Physics and Astronomy, UCLA.)\\
A high power (up to 50J) Nd-Yag laser (repetition rate 10 minutes) is focused on a target in the LAPD plasma. Measurements and simulations corroborate the generation of a collisionless shock $(M_{A})\approx 2$ across the LAPD background field in the presence of the dense, LaB$_{6}$ plasma. The interaction is studies with the use of multiple magnetic and Mach probes, fast (3 ns) photography, and spectroscopy.

\item "Laboratory Investigation of Auroral Alfv\'{e}n Electron Acceleration", C. Kletzing, F. Skiff, (Dept.\ of Physics, University of Iowa).\\
This is a study of shear Alfv\'{e}n waves with short perpendicular wavelengths as well as investigations of field-aligned acceleration of electrons due to the electric field of the waves. A series of antennas, which are phased arrays, has been developed at the University of Iowa and put on the LAPD. The propagation of waves launched by these antennas is studied and their dispersion mapped. Electron distribution functions perturbed by the Alfv\'{e}n waves are measured using a novel whistler wave diagnostic developed by the Iowa group. The results will be compared with spacecraft measurements made in the Earth's auroral region.



\end{enumerate}

\subsection{Theory-driven studies}

\begin{enumerate}
\item "Whistler Wave Pitch Angle Scattering of Electrons", Jacob Bortnick (UCLA Earth and Space Science), R.M. Thorne and Xi An (UCLA Department of Atmospheric and Oceanic Sciences)\\
This is a study of whistler wave scattering of a beam of energetic electrons. A low-density electron beam, with adjustable pitch angle relative to the background magnetic field, will form the energetic electrons. The velocity distribution function will be measured with small velocity analyzers. This will be done with and without background whistler waves. The waves will be launched with a small loop antenna. Results will be compared to theoretical predictions.

\item "Experimental Study of Alfv\'{e}n wave damping processes relevant to the solar corona"  Daniel Wolf Savin, Michael Hahn (Department of Astrophysics, Columbia University.) \\
Shear Alfv\'{e}n wave damping and heating will be studied in the context of explaining heating in solar coronal holes. The waves will be launched in magnetic field and density gradients and their propagation will be studied and wave damping evaluated in a number of scenarios. Of special interest is the propagation of waves in cross field density gradients,. The gradients will be created using grids with variable transparency across $B_{0}$ . Another area of study will be the reflection of shear Alfv\'{e}n waves in large magnetic field gradients.

\item "Experimental and Numerical Studies of Whistler Wave Ducting", A. Streltsov (Embry-Riddle Aeronautical University)\\
This study is aimed at studying the propagation of VLF whistler modes in a laboratory plasma and to compare these results with numerical predictions. A key goal is to model the propagation in magnetic field-aligned irregularities (also called channels or ducts). High frequency $(f  \ge f_{ce}/2)$ and low-frequency $(f \le f_{ce}/2)$ cases are examined.


\item "Laboratory Simulation of Magnetic Flux Rope Eruptions in the Solar Atmosphere", J. Chen, (Naval Research Laboratory.)\\
Solar flare experiments are conducted in the SMPD, the 4m, low-field plasma device. The ingredients of the flare experiment geometry are a current aligned with an arc-shaped magnetic field, together with fast ions produced by striking, simultaneously, two carbon targets with laser pulses. This arrangement is embedded in background magnetic field and plasma. The laser strike represents the eruption of a magnetic flux loop that is meant to simulate a solar coronal loop. The laser strike generates plasma flows from the foot-points of the loop that significantly modify the magnetic field topology and link the magnetic field lines of the loop with the ambient plasma. Following this event, the loop erupts by releasing its plasma into the background. The resulting impulse excites intense magnetosonic waves, that transfer energy to the ambient plasma and subsequently decay.

\item "Tearing of a Current Sheet into Magnetic Flux Ropes", W. Daughton, J. Finn (LANL), H. Karimabadi (UCSD)\\
A fully 3D kinetic code developed at Los Alamos and using the largest multiprocessor computer in the world will be used to model the tearing of a current sheet into multiple magnetic flux ropes. In full 3D computations it has been observed taht the magnetic islands, which are the result of the tearing of the current sheet are helical flux ropes which interact with one another. A new high emssivity cathode, (to be installed in the summer of 2013) will be masked to make a thin (dy/dx=20) current sheet. The full three-dimensional evolution of the current will be measured in the LAPD and detailed comparisons with theory and the petascale simulations will be done. 


%\item "Conversion of Langmuir Waves to Radio Waves", C. Cattell, P. Kellogg, Dept. of Physics, University of Minnesota.
\item "Study of Nonlinear Interaction and Turbulence of Alfv'{e}n Waves in LAPD Experiments", S. Boldyrev, J. Perez, University of Wisconsin, Madison.\\
The project is devoted to analytic and numerical study of nonlinear interaction and turbulence of Alfv'{e}n waves in the LAPD. The research is aimed at extending the results obtained for incompressible magnetohydrodynamic turbulence to plasma turbulence, and at providing analytic and numerical support to the experiments on Alfv'{e}nic turbulence conducted in LAPD. 
\end{enumerate}



\subsection{Campaigns}
The campaigns are listed as: "campaign title", "campaign leader (affiliation)"; external participants:"name (affiliation)" followed by a description. 

\begin{enumerate}

\item "Fast-Ion Campaign"\\
W. Heidbrink (UCI ); participants: M. Van Zeeland (General Atomics), B. Breizman (U.Texas, Austin), H. Boehmer (UCI), I. Furno (Lausanne). \\
An ion beam ( 25 kV , 0.5-3 A) will be injected at a variety of pitch angles into the LAPD plasma. The beam which will spiral along the magnetic field will match the phase velocity of Alfv\'{e}n waves in the background LAPD plasma. The waves are expected to be generated by Cherenkov emission from the fast ions. The goal is to create an analogue of TAE modes and study them in great detail. The helium ion beam been constructed and suvccesfully tested The project also has related side studies such as the study of the propagation of shear waves in multiple mirrors. Measurement of transport in velocity and configuration space caused by harmonic heating with compressional Alfv\'{e}n waves, resonances with shear Alfv\'{e}n waves, and drift wave turbulence.

\item "Auroral Physics Campaign" \\
 M. Koepke (West Virginia University); participants: C. Chaston (U.C. Berkeley), D. Knudsen (U. Calgary), Robert Rankin (U. Alberta).\\
 Magnetized plasmas are predicted to support electromagnetic perturbations that are static in a fixed frame if there is uniform background plasma convection. These stationary waves should not be confused with standing waves that oscillate in time with a fixed, spatially varying envelope. Stationary waves have no time variation in the fixed frame. In the drifting frame, there is an apparent time dependence as plasma convects past fixed electromagnetic structures. In this project, an off-axis, fixed channel of electron current (and depleted density) is created in the Large Plasma Device, using a small, heated, oxide-coated electrode at one plasma-column end while the larger plasma column rotates about its cylindrical axis from a radial electric field imposed by a special termination electrode on the same end. A variety of methods will be explored to generate EXB plasma flows in the center of the bulk plasma. These include segmented electrodes, spiral electrodes, emitting electrodes and a biased center conductor. The interaction will be studied with a variety of probes as well as LIF.
 
 
\item "Radiation-Belt Physics Campaign"\\
D. Papadopoulos and R. Sagdeev, University of Maryland; participants: U. Inan, T. Bell (Stanford University), S. Sharma, X. Shao (University of Maryland), W. Scales, J. Wang (VA Tech), A. Streltsov (Dartmouth).\\
The campaign is focused on the interaction of energetic electrons with launched Alfv\'{e}n and whistler waves. It is motivated by the desire to limit damage to satellites by using these waves to scatter mirror-trapped energetic electrons into the loss cone. Launching shear Alfv\'{e}n waves of arbitrary polarization was accomplished by constructing an antenna consisting of two perpendicular coils with independent phase-controlled currents. The antenna was found to launch highly collimated, relatively large amplitude shear waves with wave decay resulting mainly from collisional dissipation. The measured radiation patterns of the right-hand mode compared favorably to the predictions of an MHD simulation by the Maryland group. The second antenna studied was a classic short electric dipole. The antenna current and voltage were measured within the dipole, avoiding transmission line effects The real and imaginary parts of the antenna impedance were measured as a function of frequency and time in a decaying, afterglow plasma. A pulsed microwave source constructed for the campaign was used to inject waves at 2.45 GHz into a local magnetic mirror established in the LAPD. The fast electrons vanish when a shear wave, launched by an antenna 5 meters away is switched on. When the wave is shut off the fast electrons reappear and persist until the microwave source is pulsed off.

\item "Investigation of Sheaths near RF antennas for fusion"\\
Dan D'Ippolito (Lodestar); participants: J. Myra (MIT), J. Wright (MIT), M. Kushner (U. Michigan)\\
Study of the RF sheaths on antennas immmersed in a magnetoplasma. The antennas radiate in the ICRF, Fast Wave, regime. Antennas will be constructed at UCLA and waves launched at low and high powers into the LAPD edge plasma. A variety of probes and optical techniques will be used to study the sheath plasma waves and their coupling to fast waves and under appropriate conditions to shear Alfven waves. The experiments will be complemented with a modelling effort at Lodestar and MIT.\\

\end{enumerate}

%\subsection{Local group}


%\newpage

%\setcounter{page}{1}

%\bibliographystyle{unsrtnat}
%\bibliographystyle{prsty}
%\bibliographystyle{unsrt}
%\bibliography{refs}  




\end{document}
