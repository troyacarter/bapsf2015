\documentclass[11pt]{article}
\usepackage{lmodern}
\usepackage{amssymb,amsmath}
\usepackage{ifxetex,ifluatex}
\usepackage{fixltx2e} % provides \textsubscript
\ifnum 0\ifxetex 1\fi\ifluatex 1\fi=0 % if pdftex
  \usepackage[T1]{fontenc}
  \usepackage[utf8]{inputenc}
\else % if luatex or xelatex
  \ifxetex
    \usepackage{mathspec}
    \usepackage{xltxtra,xunicode}
  \else
    \usepackage{fontspec}
  \fi
  \defaultfontfeatures{Mapping=tex-text,Scale=MatchLowercase}
  \newcommand{\euro}{€}
\fi
% use upquote if available, for straight quotes in verbatim environments
\IfFileExists{upquote.sty}{\usepackage{upquote}}{}
% use microtype if available
\IfFileExists{microtype.sty}{%
\usepackage{microtype}
\UseMicrotypeSet[protrusion]{basicmath} % disable protrusion for tt fonts
}{}
\ifxetex
  \usepackage[setpagesize=false, % page size defined by xetex
              unicode=false, % unicode breaks when used with xetex
              xetex]{hyperref}
\else
  \usepackage[unicode=true]{hyperref}
\fi
\usepackage[usenames,dvipsnames]{color}
\hypersetup{breaklinks=true,
            bookmarks=true,
            pdfauthor={},
            pdftitle={},
            colorlinks=true,
            citecolor=blue,
            urlcolor=blue,
            linkcolor=magenta,
            pdfborder={0 0 0}}
\urlstyle{same}  % don't use monospace font for urls
\usepackage{longtable,booktabs}
\usepackage{graphicx,grffile}
\makeatletter
\def\maxwidth{\ifdim\Gin@nat@width>\linewidth\linewidth\else\Gin@nat@width\fi}
\def\maxheight{\ifdim\Gin@nat@height>\textheight\textheight\else\Gin@nat@height\fi}
\makeatother
% Scale images if necessary, so that they will not overflow the page
% margins by default, and it is still possible to overwrite the defaults
% using explicit options in \includegraphics[width, height, ...]{}
\setkeys{Gin}{width=\maxwidth,height=\maxheight,keepaspectratio}
\setlength{\parindent}{0pt}
\setlength{\parskip}{6pt plus 2pt minus 1pt}
\setlength{\emergencystretch}{3em}  % prevent overfull lines
\providecommand{\tightlist}{%
  \setlength{\itemsep}{0pt}\setlength{\parskip}{0pt}}
\setcounter{secnumdepth}{0}

\date{}

% Redefines (sub)paragraphs to behave more like sections
\ifx\paragraph\undefined\else
\let\oldparagraph\paragraph
\renewcommand{\paragraph}[1]{\oldparagraph{#1}\mbox{}}
\fi
\ifx\subparagraph\undefined\else
\let\oldsubparagraph\subparagraph
\renewcommand{\subparagraph}[1]{\oldsubparagraph{#1}\mbox{}}
\fi

\setlength{\oddsidemargin}{-0.1in}
\setlength{\topmargin}{-0.52truein} 
\setlength{\textheight}{9.15in} 
\setlength{\textwidth}{6.7in}

\usepackage[T1]{fontenc}
\usepackage{fourier}
\usepackage[sc]{mathpazo}
\linespread{1.05}         % Palatino needs more leading (space between lines)
\newcommand\Alfven{Alfv\'en }
\newcommand{\V}[1]{\mathbf{#1}} 


\usepackage{wrapfig}
\usepackage[square,numbers,sort&compress]{natbib}
\renewcommand{\cite}{\citep}
%\usepackage[psamsfonts]{amssymb}
%\usepackage{palatino}
%\usepackage{mathpazo}
%\usepackage{plasmadefs}

\hyphenation{wave-packet wave-packets}

\title{}

\begin{document}
%
\begin{center}
{\bfseries Basic Plasma Science Facility Renewal}\\
\hfill \\
T. Carter, University of California, Los Angeles (Principal Investigator)\\
W. Gekelman, University of California, Los Angeles (Co-PI)\\
G. Morales, University of California, Los Angeles (Co-PI)\\
S. Vincena, University of California, Los Angeles (Co-PI)\\
\end{center}

A renewal proposal is made to continue operation of the Basic Plasma
Science Facility (BaPSF) at the University of California, Los Angeles
(UCLA), and to support the vigorous research program of the BaPSF
scientific staff, as well as to continue the improvements in
scientific instrumentation required to maintain worldwide leadership
in fundamental plasma research. BaPSF provides national and
international scientists access to unique research devices and
diagnostic tools that permit the exploration of a wide range of
fundamental plasma problems that impact topics at the frontiers of
fusion, space science and plasma technology. The broad parameter
ranges accessible in the plasma devices operated at BaPSF allow
studies that span microscopic phenomena on the fast electron time
scales (e.g., electron plasma waves, cyclotron radiation) to the slow
time scales characteristic of plasma transport driven by drift-wave
turbulence and long wavelength magnetic fluctuations.  Qualified
researchers and research teams from universities, national
laboratories and industry can perform experiments at BaPSF, free of
charge, upon approval of their proposals by the BaPSF Scientific
Council composed of senior scientists broadly representative of the
plasma community. The BaPSF plasma devices provide effective platforms
for the training of graduate students because of their optimum,
mid-scale size. They also provide a fertile environment for the
development of junior professors by allowing them to focus entirely on
scientific research without the burden of hardware maintenance and
construction. BaPSF also operates a small, dedicated plasma device to
train high school teachers and students enrolled in the outreach
LAPTAG program.  

{\bfseries Intellectual Merit}:  The BaPSF allows the detailed
study, under controlled conditions, of fundamental questions in plasma
science that cannot be addressed in any other laboratory. The results
obtained impact a wide range of frontier topics in fusion and space
plasma research. It also lays the foundation for future developments
in plasma technology. Through creative development of plasma sources
and the operation of complementary plasma devices exploited through
focused campaigns, the operation of BaPSF constitutes a transformative
concept within plasma science.  

{\bfseries Broader Impacts}:  One of the broader
impacts of the BaPSF is that it provides unique opportunities in the
training of junior researchers from both large and small institutions,
as well as high school students and teachers. BaPSF fosters and
engenders collaborations between domestic and international
institutions and scientists from diverse areas, such as fusion, space
and industrial applications. The topical campaigns made possible by
BaPSF provides a forum for the interaction of experimentalists,
theoreticians and modelers from varied backgrounds, promoting the
cross-fertilization of ideas and techniques.

\end{document}
