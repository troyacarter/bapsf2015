\documentclass[11pt]{article}
\usepackage{lmodern}
\usepackage{amssymb,amsmath}
\usepackage{ifxetex,ifluatex}
\usepackage{fixltx2e} % provides \textsubscript
\ifnum 0\ifxetex 1\fi\ifluatex 1\fi=0 % if pdftex
  \usepackage[T1]{fontenc}
  \usepackage[utf8]{inputenc}
\else % if luatex or xelatex
  \ifxetex
    \usepackage{mathspec}
    \usepackage{xltxtra,xunicode}
  \else
    \usepackage{fontspec}
  \fi
  \defaultfontfeatures{Mapping=tex-text,Scale=MatchLowercase}
  \newcommand{\euro}{€}
\fi
% use upquote if available, for straight quotes in verbatim environments
\IfFileExists{upquote.sty}{\usepackage{upquote}}{}
% use microtype if available
\IfFileExists{microtype.sty}{%
\usepackage{microtype}
\UseMicrotypeSet[protrusion]{basicmath} % disable protrusion for tt fonts
}{}
\ifxetex
  \usepackage[setpagesize=false, % page size defined by xetex
              unicode=false, % unicode breaks when used with xetex
              xetex]{hyperref}
\else
  \usepackage[unicode=true]{hyperref}
\fi
\usepackage[usenames,dvipsnames]{color}
\hypersetup{breaklinks=true,
            bookmarks=true,
            pdfauthor={},
            pdftitle={},
            colorlinks=true,
            citecolor=blue,
            urlcolor=blue,
            linkcolor=magenta,
            pdfborder={0 0 0}}
\urlstyle{same}  % don't use monospace font for urls
\usepackage{longtable,booktabs}
\usepackage{graphicx,grffile}
\makeatletter
\def\maxwidth{\ifdim\Gin@nat@width>\linewidth\linewidth\else\Gin@nat@width\fi}
\def\maxheight{\ifdim\Gin@nat@height>\textheight\textheight\else\Gin@nat@height\fi}
\makeatother
% Scale images if necessary, so that they will not overflow the page
% margins by default, and it is still possible to overwrite the defaults
% using explicit options in \includegraphics[width, height, ...]{}
\setkeys{Gin}{width=\maxwidth,height=\maxheight,keepaspectratio}
\setlength{\parindent}{0pt}
\setlength{\parskip}{6pt plus 2pt minus 1pt}
\setlength{\emergencystretch}{3em}  % prevent overfull lines
\providecommand{\tightlist}{%
  \setlength{\itemsep}{0pt}\setlength{\parskip}{0pt}}
\setcounter{secnumdepth}{0}

\date{}

% Redefines (sub)paragraphs to behave more like sections
\ifx\paragraph\undefined\else
\let\oldparagraph\paragraph
\renewcommand{\paragraph}[1]{\oldparagraph{#1}\mbox{}}
\fi
\ifx\subparagraph\undefined\else
\let\oldsubparagraph\subparagraph
\renewcommand{\subparagraph}[1]{\oldsubparagraph{#1}\mbox{}}
\fi

\setlength{\oddsidemargin}{-0.1in}
\setlength{\topmargin}{-0.52truein} 
\setlength{\textheight}{9.15in} 
\setlength{\textwidth}{6.7in}

\usepackage[T1]{fontenc}
\usepackage{fourier}
\usepackage[sc]{mathpazo}
\linespread{1.05}         % Palatino needs more leading (space between lines)


\usepackage{wrapfig}
\usepackage[square,numbers,sort&compress]{natbib}
\renewcommand{\cite}{\citep}
%\usepackage[psamsfonts]{amssymb}
%\usepackage{palatino}
%\usepackage{mathpazo}

%\usepackage{plasmadefs}

\hyphenation{wave-packet wave-packets}

\title{}

\begin{document}


\section{Graduate students currently engaged in research using BaPSF}

\begin{tabbing}
Student \hspace{0.8in} \=  Institution/Group \hspace{2.1in} \= Mentor(s) \\
\\
Xin An \> UCLA (Atmospheric and Oceanic Sciences) \> J. Bortnick\\
Anton Bondarenko \> UCLA \> C. Niemann\\
Jeffrey Bonde \> UCLA (BaPSF group) \> W. Gekelman, S. Vincena \\
S. Eric Clark \> UCLA \> C. Niemann\\
Paul Crandall \> UCLA \> F. Jenko \\
Tim DeHaas \> UCLA (BaPSF group) \> W. Gekelman \\
Erik Everson \> UCLA \> C. Niemann\\
Daniel Guice \> UCLA (BaPSF group) \> T. Carter \\
Dooran Hong \> UCLA (BaPSF group) \> W. Gekelman\\
Mike Martin \> UCLA (BaPSF group) \> W. Gekelman, T. Carter \\
Samuel Nogami \> WVU \> M. Koepke \\
Adam Preweisch \> UC Irvine \> W. Heidbrink\\
Jeffrey Robertson \> UCLA (BaPSF group) \> T. Carter \\
Giovanni Rossi \> UCLA (BaPSF group) \> T. Carter \\
James Schroeder \> U. Iowa \> C. Kletzing, F. Skiff, G. Howes \\
\end{tabbing}


\section{Advanced degrees granted, last 5 year period}

 \begin{tabbing}
Student \hspace{0.65in} \= Degree \hspace{0.1in} \= Granting institution \hspace{1.5in} \= Graduate advisor\hspace{0.3in} \= Year \\
\\
Nathaniel Moore  \> Ph.D. \>  University of California, Los Angeles \> W. Gekelman \> 2015\\
Yiting Zhang \> Ph.D. \> University of Michigan \> M. Kushner \> 2015\\
William Farmer	 \> Ph.D. \>  University of California, Los Angeles \> G. Morales \> 2014\\
Adam Kullberg	 \> Ph.D. \>  University of California, Los Angeles \> G. Morales \> 2014\\
Derek Schaeffer  \> Ph.D. \>  University of California, Los Angeles \> C. Niemann \> 2014\\
Kris Kersten \> Ph.D. \> University of Minnesota \> C. Cattell \> 2014\\
Brett Friedman \> Ph.D. \>  University of California, Los Angeles \> T. Carter \> 2013\\
David Schaffner \> Ph.D. \>  University of California, Los Angeles \> T. Carter \> 2013\\
Yuhou Wang  \> Ph.D. \>  University of California, Los Angeles \> W. Gekelman \> 2013\\
Chris Cooper  \> Ph.D. \>  University of California, Los Angeles \> W. Gekelman \> 2012\\
David Auerbach  \> Ph.D. \>  University of California, Los Angeles \> T. Carter \> 2012\\
Shu Zhou \> PhD. \> University of California, Irvine \> W. Heidbrink
\>  2011 \\
Gregoire Hornung \> M.S. \> Gent Universiteit, Belgium \> G. Van Oost;
J. Maggs, G. Morales \> 2010 \\
Andrew Collette \> Ph.D. \>  University of California, Los Angeles \> W. Gekelman \> 2010\\
Kim De Rose\> M.S. \>  University of California, Los Angeles \> W. Gekelman \> 2010\\
Brett Jacobs \> Ph.D. \>  University of California, Los Angeles \> W. Gekelman \> 2010\\
Alexey Karavaev \> Ph.D. \>  University of Maryland \> K. Papadopoulos \> 2010\\
Nathan Kugland \> PhD. \> University of California, Los Angeles \> C. Niemann \> 2010\\
Eric Lawrence \> Ph.D. \>  University of California, Los Angeles \>
W. Gekelman \> 2010\\
Franklin Chaing \> Ph.D. \> University of California, Los Angeles \>
J. Judy
\> 2010 \\
 \end{tabbing}
 
\section{Current active external user groups}

%\subsection{External users}
%Currently, there are thirteen active external user groups of the facility: four independent experimenter user groups, six theory-driven studies and three topical campaigns.  The four experimental groups and six-theory driven studies are listed below as: ?project name?, ?research team? (leader listed first), ?affiliation?.


\subsection{Independent experimenter user groups}

\begin{enumerate}

\item ``Laser Driven shock waves in the LAPD'', C. Niemann,
  C. Constantin (Dept. of Physics and Astronomy,
  UCLA.)\\ High power lasers are used to drive collisionless
  magnetized shocks in LAPD.  A high power (up to 50J) Nd-Yag laser (repetition rate 10
  minutes) is focused on a target in the LAPD plasma. Measurements and
  simulations corroborate the generation of a collisionless shock
  $(M_{A})\approx 2$ across the LAPD background field in the presence
  of the dense, LaB$_{6}$ plasma. The interaction is studies with the
  use of multiple magnetic and Mach probes, fast (3 ns) photography,
  and spectroscopy.

\item ``Laboratory Investigation of Auroral Alfv\'{e}n Electron
  Acceleration'', C. Kletzing, F. Skiff, (Dept. of Physics, University
  of Iowa).\\ This is a study of shear Alfv\'{e}n waves with short
  perpendicular wavelengths as well as investigations of field-aligned
  acceleration of electrons due to the electric field of the waves. A
  series of antennas, which are phased arrays, has been developed at
  the University of Iowa and put on the LAPD. The propagation of waves
  launched by these antennas is studied and their dispersion
  mapped. Electron distribution functions perturbed by the Alfv\'{e}n
  waves are measured using a novel whistler wave diagnostic developed
  by the Iowa group. The results will be compared with spacecraft
  measurements made in the Earth's auroral region.

\item ``Development and testing of LaB$_6$ sources for Tri-Alpha
  Energy,'' D. Bui, A. Song (Tri Alpha)

\item ``Development of plasma microprobes,''  J. Judy (UCLA EE).

\end{enumerate}

\subsection{Theory-driven studies}

\begin{enumerate}
\item Colestock?

\item ``Whistler Wave Pitch Angle Scattering of Electrons'', Jacob
  Bortnick (UCLA Earth and Space Science), R.M. Thorne and Xi An (UCLA
  Department of Atmospheric and Oceanic Sciences)\\ This is a study of
  whistler wave scattering of a beam of energetic electrons. A
  low-density electron beam, with adjustable pitch angle relative to
  the background magnetic field, will form the energetic
  electrons. The velocity distribution function will be measured with
  small velocity analyzers. This will be done with and without
  background whistler waves. The waves will be launched with a small
  loop antenna. Results will be compared to theoretical predictions.

\item ``Tearing of a Current Sheet into Magnetic Flux Ropes'',
  W. Daughton, J. Finn (LANL), H. Karimabadi (UCSD)\\ A fully 3D
  kinetic code developed at Los Alamos and using the largest
  multiprocessor computer in the world will be used to model the
  tearing of a current sheet into multiple magnetic flux ropes. In
  full 3D computations it has been observed taht the magnetic islands,
  which are the result of the tearing of the current sheet are helical
  flux ropes which interact with one another. A new high emissivity
  cathode, (installed in the summer of 2013) will be masked to
  make a thin (dy/dx=20) current sheet. The full three-dimensional
  evolution of the current will be measured in the LAPD and detailed
  comparisons with theory and the petascale simulations will be done.


\item Kushner?


\item ``Investigation of Sheaths near RF antennas for fusion''\\ D.
  D'Ippolito, J. Myra (Lodestar)\\ Study of the RF sheaths on antennas
  immmersed in a magnetoplasma. The antennas radiate in the ICRF, Fast
  Wave, regime. Antennas will be constructed at UCLA and waves
  launched at low and high powers into the LAPD edge plasma. A variety
  of probes and optical techniques will be used to study the sheath
  plasma waves and their coupling to fast waves and under appropriate
  conditions to shear Alfven waves. The experiments will be
  complemented with a modeling effort at Lodestar.\\

\item ``Experimental and Numerical Studies of Whistler Wave Ducting,''
  A. Streltsov (Embry-Riddle Aeronautical University)\\ This study is
  aimed at studying the propagation of VLF whistler modes in a
  laboratory plasma and to compare these results with numerical
  predictions. A key goal is to model the propagation in magnetic
  field-aligned irregularities (also called channels or ducts). High
  frequency $(f \ge f_{ce}/2)$ and low-frequency $(f \le f_{ce}/2)$
  cases are examined.

%% \item "Laboratory Simulation of Magnetic Flux Rope Eruptions in the
%%   Solar Atmosphere", J. Chen, (Naval Research Laboratory.)\\ Solar
%%   flare experiments are conducted in the SMPD, the 4m, low-field
%%   plasma device. The ingredients of the flare experiment geometry are
%%   a current aligned with an arc-shaped magnetic field, together with
%%   fast ions produced by striking, simultaneously, two carbon targets
%%   with laser pulses. This arrangement is embedded in background
%%   magnetic field and plasma. The laser strike represents the eruption
%%   of a magnetic flux loop that is meant to simulate a solar coronal
%%   loop. The laser strike generates plasma flows from the foot-points
%%   of the loop that significantly modify the magnetic field topology
%%   and link the magnetic field lines of the loop with the ambient
%%   plasma. Following this event, the loop erupts by releasing its
%%   plasma into the background. The resulting impulse excites intense
%%   magnetosonic waves, that transfer energy to the ambient plasma and
%%   subsequently decay.



\item ``Search for electron solitary structures,'' L.-J. Chen
  (University of New Hampshire)\\
This project is motivated by the ubiquitous observation made on board
spacecraft of electrostatic solitary structures known as ``electron
holes''. The major outstanding questions are related to the generation,
dynamics and statistics of phase-space structures of spatial dimension
comparable to the Debye length. These features have been investigated
by injecting a small suprathermal electron beam into the LAPD plasma
and measuring the small structures with novel MEMS microscopic probes
that sample the structures at rates much higher than the plasma
frequency. The measured scales and amplitudes of these structures are
comparable to those derived from observation in the
magnetosphere. However, the measured velocities indicate that they are
not generated by an instability driven by the initially injected
beam. Instead, the solitary structures have the same scales and
propagate at the same speed as coherent wave packets and background
fluctuations that are  identified as electrostatic
whistler waves in a strongly Landau damped regime.


\item ``Experimental study of Alfv\'{e}n wave damping processes
  relevant to the solar corona,'' Daniel Wolf Savin, Michael Hahn
  (Columbia University) \\ Shear
  Alfv\'{e}n wave damping and heating will be studied in the context
  of explaining heating in solar coronal holes. The waves will be
  launched in magnetic field and density gradients and their
  propagation will be studied and wave damping evaluated in a number
  of scenarios. Of special interest is the propagation of waves in
  cross-field density gradients,. The gradients will be created using
  grids with variable transparency across $B_{0}$ . Another area of
  study will be the reflection of shear Alfv\'{e}n waves in large
  magnetic field gradients.

%\item "Conversion of Langmuir Waves to Radio Waves", C. Cattell, P. Kellogg, Dept. of Physics, University of Minnesota.
%% \item "Study of Nonlinear Interaction and Turbulence of Alfv\'{e}n
%%   Waves in LAPD Experiments", S. Boldyrev, J. Perez, University of
%%   Wisconsin, Madison.\\ The project is devoted to analytic and
%%   numerical study of nonlinear interaction and turbulence of Alfv\'{e}n
%%   waves in the LAPD. The research is aimed at extending the results
%%   obtained for incompressible magnetohydrodynamic turbulence to plasma
%%   turbulence, and at providing analytic and numerical support to the
%%   experiments on Alfv\'{e}nic turbulence conducted in LAPD.



\end{enumerate}



\subsection{Campaigns}
The campaigns are listed as: "campaign title", "campaign leader (affiliation)"; external participants:"name (affiliation)" followed by a description. 

\begin{enumerate}

\item "Fast-Ion Campaign"\\ W. Heidbrink (UCI ); participants: M. Van
  Zeeland (General Atomics), B. Breizman (U.Texas, Austin), H. Boehmer
  (UCI), I. Furno (Lausanne), F. Jenko (MPI/UCLA), S. Tripathi,
  S. Vincena, T. Carter (UCLA) \\ An ion beam ( 25 kV , 0.5-3 A) will
  be injected at a variety of pitch angles into the LAPD plasma. The
  beam which will spiral along the magnetic field will match the phase
  velocity of Alfv\'{e}n waves in the background LAPD plasma. The
  waves are expected to be generated by Cherenkov emission from the
  fast ions. The goal is to create an analogue of TAE modes and study
  them in great detail. The helium ion beam been constructed and
  suvccesfully tested The project also has related side studies such
  as the study of the propagation of shear waves in multiple
  mirrors. Measurement of transport in velocity and configuration
  space caused by harmonic heating with compressional Alfv\'{e}n
  waves, resonances with shear Alfv\'{e}n waves, and drift wave
  turbulence.

  "Study of Ion Transport in Turbulent Plasmas", W. Heidbrink,
  R. McWilliams, H. Boehmer (Dept. of Physics, University of
  California, Irvine.)\\ Continuation of experiments investigating the
  interaction between fast ions and waves and turbulence in LAPD.  A
  moderate energy ($\sim 1$~keV), low current Lithium ion beam is
  mounted in the LAPD. The beam provides a source of test ions, whose
  trajectories can be

spirals along the background
  magnetic field in an argon or helium plasma. The beam profile will
  be measured with probes as it moves through localized turbulent
  layers. The layers are generated with antennas. The beam divergence
  and energy spread is being studied.


\item "Auroral Physics Campaign" \\ M. Koepke (West Virginia
  University); participants: C. Chaston (U.C. Berkeley), D. Knudsen
  (U. Calgary), Robert Rankin (U. Alberta), S. Vincena, W. Gekelman
  (UCLA) \\ Magnetized plasmas are
  predicted to support electromagnetic perturbations that are static
  in a fixed frame if there is uniform background plasma
  convection. These stationary waves should not be confused with
  standing waves that oscillate in time with a fixed, spatially
  varying envelope. Stationary waves have no time variation in the
  fixed frame. In the drifting frame, there is an apparent time
  dependence as plasma convects past fixed electromagnetic
  structures. In this project, an off-axis, fixed channel of electron
  current (and depleted density) is created in the Large Plasma
  Device, using a small, heated, oxide-coated electrode at one
  plasma-column end while the larger plasma column rotates about its
  cylindrical axis from a radial electric field imposed by a special
  termination electrode on the same end. A variety of methods will be
  explored to generate EXB plasma flows in the center of the bulk
  plasma. These include segmented electrodes, spiral electrodes,
  emitting electrodes and a biased center conductor. The interaction
  will be studied with a variety of probes as well as LIF.
 
 
\item "Radiation-Belt Physics Campaign"\\ D. Papadopoulos and
  R. Sagdeev, University of Maryland; participants: U. Inan, T. Bell
  (Stanford University), S. Sharma, X. Shao (University of Maryland),
  W. Scales, J. Wang (VA Tech), A. Streltsov (Dartmouth), Y. Wang,
  W. Gekelman (UCLA).\\ The
  campaign is focused on the interaction of energetic electrons with
  launched Alfv\'{e}n and whistler waves. It is motivated by the
  desire to limit damage to satellites by using these waves to scatter
  mirror-trapped energetic electrons into the loss cone. Launching
  shear Alfv\'{e}n waves of arbitrary polarization was accomplished by
  constructing an antenna consisting of two perpendicular coils with
  independent phase-controlled currents. The antenna was found to
  launch highly collimated, relatively large amplitude shear waves
  with wave decay resulting mainly from collisional dissipation. The
  measured radiation patterns of the right-hand mode compared
  favorably to the predictions of an MHD simulation by the Maryland
  group. The second antenna studied was a classic short electric
  dipole. The antenna current and voltage were measured within the
  dipole, avoiding transmission line effects The real and imaginary
  parts of the antenna impedance were measured as a function of
  frequency and time in a decaying, afterglow plasma. A pulsed
  microwave source constructed for the campaign was used to inject
  waves at 2.45 GHz into a local magnetic mirror established in the
  LAPD. The fast electrons vanish when a shear wave, launched by an
  antenna 5 meters away is switched on. When the wave is shut off the
  fast electrons reappear and persist until the microwave source is
  pulsed off.


\end{enumerate}


\section{Publications in refereed journals (funding cycle 2010 - mid 2015)}
 \begin{enumerate}

\item S. Dorfman, T.A. Carter, Non-linear Alfv\'{e}n wave interaction
  leading to resonant excitation of an acoustic mode in the
  laboratory, Phys. Plasmas, 22, 055706 (2015);
  http://dx.doi.org/10.1063/1.4919275

\item B. Friedman and T.A. Carter, A non-modal analytical method to
  predict turbulent properties applied to the Hasegawa-Wakatani model,
  Phys. Plasmas, 22, 012307 (2015); http://dx.doi.org/10.1063/1.4905863

\item B. Van Compernolle, X. An, J. Bortnik, R.?M. Thorne, P. Pribyl, and W. Gekelman, Excitation of Chirping Whistler Waves in a Laboratory Plasma, Phys. Rev. Lett. 114, 245002 (2015) http://dx.doi.org/10.1103/PhysRevLett.114.245002

\item M. J. Martin, J. Bonde, W. Gekelman, and P. Pribyl, A resistively heated CeB6 emissive probe, Rev. Sci. Instrum., 86, 053507 (2015) [http://dx.doi.org/10.1063/1.4921838]

\item B. Van Compernolle, G. J. Morales, J. E. Maggs, and R. D. Sydora, Laboratory study of avalanches in magnetized plasmas, Phys. Rev. E 91, 031102(R) (2015). http://dx.doi.org/10.1103/PhysRevE.91.031102 .

\item J. E. Maggs, T.L. Rhodes, and G.J. Morales, Chaotic density fluctuations in L-mode plasmas of the DIII-D tokamak, Plasma Phys. Control. Fusion 57 045004 (2015) http://dx.doi.org/10.1088/0741-3335/57/4/045004

\item S. K. P. Tripathi, B. Van Compernolle, W. Gekelman, P. Pribyl, and W. Heidbrink, Excitation of shear Alfv\'{e}n waves by a spiraling ion beam in a large magnetoplasma, Phys. Rev. E v91, 013109 (2015) http://dx.doi.org/10.1103/PhysRevE.91.013109

\item   A. S. Bondarenko, D. B. Schaeffer, E. T. Everson, S. E. Clark, C. G. Constantin, and C. Niemann, Spectroscopic measurement of high-frequency electric fields in the interaction of explosive debris plasma with magnetized background plasma, Phys. Plasmas, v21, 122112 (2014). [DOI: 10.1063/1.4904374]

\item   S. E. Clark, E. T. Everson, D. B. Schaeffer, A. S. Bondarenko, C. G. Constantin, C. Niemann, and D. Winske, Enhanced collisionless shock formation in a magnetized plasma containing a density gradient, Phys. Rev. E, v90, 041101(R) (2014), DOI: 10.1103/PhysRevE.90.041101

\item    C. Niemann,W. Gekelman, C. G. Constantin, E. T. Everson, D. B. Schaeffer, A. S. Bondarenko, S. E. Clark, D.Winske, S. Vincena, B. Van Compernolle, and P. Pribyl, Observation of collisionless shocks in a large current-free laboratory plasma, Geophys. Res. Lett., 41, doi:10.1002/2014GL061820 (2014)

\item   D. B. Schaeffer, E. T. Everson, A. S. Bondarenko, S. E. Clark, C. G. Constantin, S. Vincena, B. Van Compernolle, S. K. P. Tripathi, D. Winske, W. Gekelman, and C. Niemann, Laser-driven, magnetized quasi-perpendicular collisionless shocks on the Large Plasma Device, Phys. Plasmas, 21, 056312 (2014) [http://dx.doi.org/10.1063/1.4876608]

\item  Wang, Y. and Gekelman, W. and Pribyl, P. and Papadopoulos, K., Enhanced loss of magnetic-mirror-trapped fast electrons by a shear Alfv\'{e}n wave, Physics of Plasmas (1994-present), 21, 055705 (2014), DOI:http://dx.doi.org/10.1063/1.4874332

\item   B. Van Compernolle, J. Bortnik, P. Pribyl, W. Gekelman, M. Nakamoto, X. Tao, and R.?M. Thorne, Direct Detection of Resonant Electron Pitch Angle Scattering by Whistler Waves in a Laboratory Plasma, Phys. Rev. Lett. 112, 145006 ? Published 10 April 2014 , http://dx.doi.org/10.1103/PhysRevLett.112.145006

\item  Walter Gekelman, Bart Van Compernolle, Tim DeHaas and Stephen Vincena, Chaos in magnetic flux ropes, Plasma Phys. Control. Fusion 56 (2014) 064002 (18pp), doi:10.1088/0741-3335/56/6/064002

\item   Yiting Zhang, Mark J. Kushner, Nathaniel Moore, Patrick Pribyl, and Walter Gekelman, Space and phase resolved ion energy and angular distributions in single- and dual-frequency capacitively coupled plasmas, J. Vac. Sci. Technol. A 31(6), Nov/Dec 2013, [http://dx.doi.org/10.1116/1.4822100]

\item  D. J. Drake, J. W. R. Schroeder, G. G. Howes, C. A. Kletzing, F. Skiff, T. A. Carter, and D. W. Auerbach, Alfv\'{e}n wave collisions, the fundamental building block of plasma turbulence. IV. Laboratory experiment, Phys. Plasmas 20, 072901 (2013); http://dx.doi.org/10.1063/1.4813242

\item  G. G. Howes, K. D. Nielson, D. J. Drake, J. W. R. Schroeder, F. Skiff, C. A. Kletzing, and T. A. Carter, Alfv\'{e}n wave collisions, the fundamental building block of plasma turbulence. III. Theory for experimental design, Phys. Plasmas 20, 072304 (2013); http://dx.doi.org/10.1063/1.4812808

\item  W. A. Farmer and G. J. Morales, Propagation of shear Alfv\'{e}n waves in two-ion species plasmas confined by a nonuniform magnetic field, Phys. Plasmas 20, 082132 (2013); http://dx.doi.org/10.1063/1.4819776

\item   Nathaniel B. Moore, Walter Gekelman Patrick Pribyl, Yiting Zhang, and Mark J. Kushner, 2-dimensional ion velocity distributions measured by laser-induced fluorescence above a radio-frequency biased silicon wafer, Phys. Plasmas, 20, 083506 (2013) DOI: [http://dx.doi.org/10.1063/1.4817275]

\item  C.M. Cooper and W. Gekelman, Termination of a Magnetized Plasma on a Neutral Gas: The End of the Plasma, Phys. Rev. Lett., 110, 265001 (2013), DOI: 10.1103/PhysRevLett.110.265001

\item   J. E. Maggs and G. J. Morales, Permutation entropy analysis of temperature fluctuations from a basic electron heat transport experiment, Plasma Phys. Control. Fusion 55 (2013) 085015 (7pp), doi:10.1088/0741-3335/55/8/085015

\item  Y. Wang, W. Gekelman, and P. Pribyl, Hard x-ray tomographic studies of the destruction of an energetic electron ring, Rev. Sci. Instrum., v84, 053503 (2013) ; DOI:10.1063/1.4804354

\item   D. A. Schaffner, T. A. Carter, G. D. Rossi, D. S. Guice, J. E. Maggs, S. Vincena, and B. Friedman, Turbulence and transport suppression scaling with flow shear on the Large Plasma Device, Phys. Plasmas 20, 055907 (2013); DOI: http://dx.doi.org/10.1063/1.4804637

\item   B. Friedman, T. A. Carter, M. V. Umansky, D. Schaffner, and I. Joseph, Nonlinear instability in simulations of Large Plasma Device turbulence, Phys. Plasmas 20, 055704 (2013); DOI: 10.1063/1.4805084

\item  S. Dorfman and T. A. Carter, Nonlinear Excitation of Acoustic Modes by Large-Amplitude Alfv\'{e}n Waves in a Laboratory Plasma, Phys. Rev. Lett., 110, 195001 (2013) DOI: 10.1103/PhysRevLett.110.195001

\item  S.K.P. Tripathi and W. Gekelman, Dynamics of an Erupting Arched Magnetic Flux Rope in a Laboratory Plasma Experiment, Solar Phys., 0038-0938 (2013) DOI: 10.1007/s11207-013-0257-0

\item   S. T. Vincena, W. A. Farmer, J. E. Maggs, and G. J. Morales, Investigation of an ion-ion hybrid Alfv\'{e}n wave resonator, Phys. Plasmas, 20, 012110 (2013) http://dx.doi.org/10.1063/1.4775777.

\item   C. Niemann, W. Gekelman, C. G. Constantin, E. T. Everson, D. B. Schaeffer, S. E. Clark, D. Winske, A. B. Zylstra, P. Pribyl, S. K. P. Tripathi, D. Larson, S. H. Glenzer, and A. S. Bondarenk, Dynamics of exploding plasmas in a large magnetized plasma, Phys. Plasmas 20, 012108 (2013). [http://dx.doi.org/10.1063/1.4773911]

\item   G. G. Howes, D. J. Drake, K. D. Nielson, T. A. Carter, C. A. Kletzing, and F. Skiff, Toward astrophysical turbulence in the laboratory, Phys. Rev. Lett. 109, 255001 (2012); http://dx.doi.org/10.1103/PhysRevLett.109.255001.

\item   J.E. Maggs and G.J. Morales, Exponential power spectra, deterministic chaos and Lorentzian pulses in plasma edge dynamics, Plasma Phys. Control. Fusion, 54, 124041 (2012) doi:10.1088/0741-3335/54/12/124041

\item W W Heidbrink, H Boehmer, R McWilliams, A Preiwisch, Y Zhang, L Zhao, S Zhou, A Bovet, A Fasoli, I Furno, K Gustafson, P Ricci, T Carter, D Leneman, S K P Tripathi, and S Vincena, Measurements of interactions between waves and energetic ions in basic plasma experiments, Plasma Phys. Control. Fusion, v54, 124007 (2012); doi:10.1088/0741-3335/54/12/124007

\item  B. Friedman, T. A. Carter, M. V. Umansky, D. Schaffner, and B. Dudson, Energy dynamics in a simulation of LAPD turbulence, Phys. Plasmas, 102307 (2012); DOI: 10.1063/1.4759010.

\item  C. Niemann, C.G. Constantin, D.B. Schaeffer, A. Tauschwitz, T. Weiland, Z. Lucky, W. Gekelman, E.T. Everson, and D. Winske, High-energy Nd:glass laser facility for collisionless laboratory astrophysics, JINST 7 P03010 (2012);.doi:10.1088/1748-0221/7/03/P03010

\item B. Van Compernolle and W. Gekelman, Morphology and dynamics of three interacting kink-unstable flux ropes in a laboratory magnetoplasma, Phys. Plasmas 19, 102102 (2012); http://dx.doi.org/10.1063/1.4755949.

\item D. A. Schaffner, T. A Carter, G. D. Rossi, D. S. Guice, J. E. Maggs, S. Vincena, and B. Friedman, Modification of Turbulent Transport with Continuous Variation of Flow Shear in the Large Plasma Device, Phys. Rev. Lett. 109, 135002 (2012); DOI: 10.1103/PhysRevLett.109.135002.

\item J.E. Maggs and G.J. Morales, Origin of Lorentzian pulses in deterministic chaos, Phys. Rev. E 86, 015401(R) (2012) DOI: 10.1103/PhysRevE.86.015401

\item  W..A. Farmer and G.J. Morales, Cherenkov radiation of shear Alfv\'{e}n waves in plasmas with two ion species, Phys Plasmas, 19, 092109 (2012); [http://dx.doi.org/10.1063/1.4751462]

\item  W. Gekelman, E. Lawrence, and B. Van Compernolle, Three-dimensional reconnection involving magnetic flux ropes, ApJ, 753:131, (2012); [doi:10.1088/0004-637X/753/2/131].

\item  A. V. Streltsov, J. Woodroffe, W. Gekelman, and P. Priby, Modeling the propagation of whistler-mode waves in the presence of field-aligned density irregularities, Phys Plasmas 19, 052104 (2012)l [http://dx.doi.org/10.1063/1.4719710].

\item  Zhou, Shu, W.W. Heidbrink, H. Boehmer, R. McWilliams, T.A. Carter, S. Vincena, S.K.P. Tripathi, and B. Van Compernolle, Thermal plasma and fast ion transport in electrostatic turbulence in the large plasma device, Phys. Plasmas, 19, 055904 (2012); [http://dx.doi.org/10.1063/1.3695341].

\item  Yuhou Wang, Walter Gekelman, Patrick Pribyl, and Konstantinos Papadopoulos, Scattering of Magnetic Mirror Trapped Fast Electrons by a Shear Alfv\'{e}n Wave, Phys. Rev. Lett. 108, 105002 (2012); [DOI: 10.1103/PhysRevLett.108.105002].

\item   Zhou, S., Heidbrink, W.W., Boehmer, H., McWilliams, R., Carter, T.A., Vincena, S., Friedman, B., and Schaffner, D., Sheared-flow induced confinement transition in a linear magnetized plasma, Phys. Plasmas, v19, 012116 (2012); [doi:10.1063/1.3677361].

\item   B. Van Compernolle, W. Gekelman, P. Pribyl, and C. M. Cooper, Wave and transport studies utilizing dense plasma filaments generated with a lanthanum hexaboride cathode, Phys. Plasmas, v18, 123501 (2011); [doi:10.1063/1.3671909].

\item  Maggs J. E.; Morales G. J., Generality of Deterministic Chaos, Exponential Spectra, and Lorentzian Pulses in Magnetically Confined Plasmas, Phys. Rev. Lett. 107, 185003 (2011); DOI: 10.1103/PhysRevLett.107.185003.

\item  D. J. Drake, C. A. Kletzing, F. Skiff, G. G. Howes, and S. Vincena, Design and use of an Elasser probe for analysis of Alfv\'{e}n wave fields according to wave direction, Rev. Sci. Instrum., v82 103505 (2011); [doi:10.1063/1.3649950].

\item   S. K. P. Tripathi, P. Pribyl, and W. Gekelman, Development of a radio-frequency ion beam source for fast-ion studies on the large plasma device, Rev. Sci. Instrum. 82, 093501 (2011); [doi:10.1063/1.3631628].

\item   W. Gekelman, P. Pribyl, J. Wise, A. Lee, R. Hwang, C. Eghtebas, J. Shin, and B. Baker, Using plasma experiments to illustrate a complex index of refraction, Am. J. Phys. 79 (9), September 2011; http://dx.doi.org/10.1119/1.3591341

\item   S. K. P. Tripathi and W. Gekelman, Laboratory simulation of solar magnetic flux rope eruptions, Proceedings of the International Astronomical Union 01 August 2010 6: 483-486, DOI: http://dx.doi.org/10.1017/S1743921311015845.

\item   G. Hornung, B. Nold, J. E. Maggs, G. J. Morales, M. Ramisch, and U. Stroth, Observation of exponential spectra and Lorentzian pulses in the TJ-K stellarator, Phys. Plasmas 18, 082303 (2011); doi:10.1063/1.3622679.

\item  Shu Zhou, W. W. Heidbrink, H. Boehmer, R. McWilliams, T. A. Carter, S. Vincena, and S. K. P. Tripathi, Dependence of fast-ion transport on the nature of the turbulence in the Large Plasma Device, Phys. Plasmas 18, 082104 (2011); doi:10.1063/1.3622203.

\item  Chiang, F.C., Pribyl, P., Gekelman, W., Lefebvre, B., Chen, Li-Jen, and Judy, J. W. Microfabricated Flexible Electrodes for Multiaxis Sensing in the Large Plasma Device at UCLA, IEEE Trans. Plasma Sci. v39, n6, June (2011); doi:10.1109/TPS.2011.2129601.

\item  Gekelman, W., Vincena, S., Van Compernolle, B., Morales, G.J., Maggs, J.E., Pribyl, P., and Carter, T.A., The many faces of shear Alfv\'{e}n waves, Phys. Plasmas, v18, 055501, (2011); doi:10.1063/1.3592210.

\item   Vincena, S. T., W. A. Farmer, J. E. Maggs, and G. J. Morales (2011), Laboratory realization of an ion-ion hybrid Alfv\'{e}n wave resonator, Geophys. Res. Lett., 38, L11101, doi:10.1029/2011GL047399.

\item  B. Jacobs, W. Gekelman, P. Pribyl, and M. Barnes, Temporally resolved ion velocity distribution measurements in a radio-frequency plasma sheath, Phys. Plasmas, v18, 053503 (2011); [doi:10.1063/1.3577575].

\item  A. Collette and W. Gekelman, Structure of an exploding laser-produced plasma, Phys. Plasmas, v18, 055705 (2011); [doi:10.1063/1.3567525].

\item   Auerbach, D.W., T.A. Carter, S. Vincena, and P. Popovich, Resonant drive and nonlinear suppression of gradient-driven instabilities via interaction with shear Alfv\'{e}n waves, Phys. Plasmas, v18, 055708 (2011) [doi:10.1063/1.3574506].

\item Umansky M. V.; Popovich P.; Carter T. A.; Friedman B.; Nevins W. M., Numerical simulation and analysis of plasma turbulence the Large Plasma Device, Phys. Plasmas v18, 055709 (2011); [doi:10.1063/1.3567033].

\item  A. V. Karavaev, N. A. Gumerov, K. Papadopoulos, Xi Shao, A. S. Sharma, W. Gekelman, Y. Wang, B. Van Compernolle, P. Pribyl, and S. Vincena, Generation of shear Alfv\'{e}n waves by a rotating magnetic field source: Three-dimensional simulations, Phys. Plasmas, Phys. Plasmas, v18, 032113, 2011; DOI:10.1063/1.3562118.

\item  B. Lefebvre, L.-J. Chen, W. Gekelman, P. Kintner, J. Pickett, P. Pribyl, and S. Vincena, Debye-scale solitary structures measured in a beam-plasma laboratory experiment, Nonlin. Process. Geophys., 18, 41-47, 2011; doi:10.5194/npg-18-41-2011.

\item   P. Popovich, M.V. Umansky, T.A. Carter, and B. Friedman, Modeling plasma turbulence and transport in the Large Plasma Device, Phys. Plasmas, v17, 122312, 2010; doi:10.1063/1.3527987.

\item Shu Zhou, W. W. Heidbrink, H. Boehmer, R. McWilliams, T. Carter, S. Vincena, S. K. P. Tripathi, P. Popovich, B. Friedman, and F. Jenko, Turbulent transport of fast ions in the Large Plasma Device, Phys. Plasmas, v17 092103, (2010); [doi:10.1063/1.3486532].

\item  W. Gekelman, E. Lawrence, A. Collette, S. Vincena, B. VanCompernolle, P. Pribyl, M. Berger, and J. Campbell, Magnetic field line reconnection in the current systems of flux ropes and Alfv\'{e}n, Phys. Scr. T142 014032 (2010); doi:10.1088/0031-8949/2010/T142/014032.

\item  P. Pribyl, W. Gekelman, and A. Gigliotti, Direct measurement of the radiation resistance of a dipole antenna in the whistler/lower hybrid wave regime, Radio Sci., v45, RS4013, (2010); DOI: 10.1029/2009RS004266.

\item  D. B. Schaeffer, N. L. Kugland, C. G. Constantin, E. T. Everson, B. Van Compernolle, C. A. Ebbers, S. H. Glenzer, and C. Niemann, A scalable multipass laser cavity based on injection by frequency conversion for noncollective Thomson scattering, Rev. Sci. Instrum. 81, 10D518 (2010).  http://dx.doi.org/10.1063/1.3460626

\item  Collette, A. and Gekelman, W., Structure of an exploding laser-produced plasma, Phys. Rev. Lett., 105, 195003 (2010); DOI:10.1103/PhysRevLett.105.195003.

\item  Auerbach, D.W., Carter, T.A., Vincena, S., and Popovich, P., Control of gradient-driven instabilities using shear Alfv\'{e}n beat waves, Phys. Rev. Lett., 105, 13505 (2010); DOI: 10.1103/PhysRevLett.105.135005.

\item  B. Lefebvre, L. Chen, W. Gekelman, P. Kintner, J. Pickett, P. Pribyl, S. Vincena, F.Chiang, and J. Judy, Laboratory measurements of electrostatic solitary structures generated by beam injection, Phys. Rev. Lett., v105, 115001, (2010); DOI: 10.1103/PhysRevLett.105.115001.

\item  C.M. Cooper, W. Gekelman, P. Pribyl, and Z. Lucky, A new large area lanthanum hexaboride plasma source, Rev. Sci. Instrum, 81, 083503 (2010) 105, 075005 (2010); doi:10.1063/1.3471917.

\item   S.K.P. Tripathi and W. Gekelman, Laboratory Simulation of Arched Magnetic Flux Rope Eruptions in the Solar Atmosphere, Phys. Rev. Lett. 105, 075005 (2010); DOI: 10.1103/PhysRevLett.105.075005.

\item   B. Jacobs, W. Gekelman, P. Pribyl, and M. Barnes, Phase-Resolved Measurements of Ion Velocity in a Radio-Frequency Sheath, Phys. Rev. Lett., v105, 075001, (2010); DOI: 10.1103/PhysRevLett.105.075001.

\item   S.T. Vincena, G.J. Morales, and J.E. Maggs, Effect of two ion species on the propagation of shear Alfv\'{e}n waves of small transverse scale, Phys. Plasmas, v17, 052106 (2010); DOI: 10.1063/1.3422549.

\item   C. A. Kletzing, D. J. Thuecks, F. Skiff, S. R. Bounds, and S. Vincena, Measurements of Inertial Limit Alfv\'{e}n Wave Dispersion for Finite Perpendicular Wave Number, Phys. Rev. Lett. 104, 095001 (2010); DOI: 10.1103/PhysRevLett.104.095001.

\item  A.V. Karavaev, N.A. Gumerov, K. Papadopoulos, Xi Shao, A.S. Sharma, W. Gekelman, A. Gigliotti, P. Pribyl, and S. Vincena, Generation of whistler waves by a rotating magnetic field source, Phys. Plasmas, v17, 012102, 2010. http://dx.doi.org/10.1063/1.3274916

\item  A. B. Zylstra, C. Constantin, E. T. Everson, D. Schaeffer, N. L. Kugland, P. Pribyl, and C. Niemanna, Ion velocity distribution measurements in a magnetized laser plasma expansion, JINST 5 P06004 (2010); doi:10.1088/1748-0221/5/06/P06004.

\end{enumerate}
	








\end{document}
