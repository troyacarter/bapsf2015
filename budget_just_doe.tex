\documentclass[11pt]{article}
\usepackage{lmodern}
\usepackage{amssymb,amsmath}
\usepackage{ifxetex,ifluatex}
\usepackage{fixltx2e} % provides \textsubscript
\ifnum 0\ifxetex 1\fi\ifluatex 1\fi=0 % if pdftex
  \usepackage[T1]{fontenc}
  \usepackage[utf8]{inputenc}
\else % if luatex or xelatex
  \ifxetex
    \usepackage{mathspec}
    \usepackage{xltxtra,xunicode}
  \else
    \usepackage{fontspec}
  \fi
  \defaultfontfeatures{Mapping=tex-text,Scale=MatchLowercase}
  \newcommand{\euro}{€}
\fi
% use upquote if available, for straight quotes in verbatim environments
\IfFileExists{upquote.sty}{\usepackage{upquote}}{}
% use microtype if available
\IfFileExists{microtype.sty}{%
\usepackage{microtype}
\UseMicrotypeSet[protrusion]{basicmath} % disable protrusion for tt fonts
}{}
\ifxetex
  \usepackage[setpagesize=false, % page size defined by xetex
              unicode=false, % unicode breaks when used with xetex
              xetex]{hyperref}
\else
  \usepackage[unicode=true]{hyperref}
\fi
\usepackage[usenames,dvipsnames]{color}
\hypersetup{breaklinks=true,
            bookmarks=true,
            pdfauthor={},
            pdftitle={},
            colorlinks=true,
            citecolor=blue,
            urlcolor=blue,
            linkcolor=magenta,
            pdfborder={0 0 0}}
\urlstyle{same}  % don't use monospace font for urls
\usepackage{longtable,booktabs}
\usepackage{graphicx,grffile}
\makeatletter
\def\maxwidth{\ifdim\Gin@nat@width>\linewidth\linewidth\else\Gin@nat@width\fi}
\def\maxheight{\ifdim\Gin@nat@height>\textheight\textheight\else\Gin@nat@height\fi}
\makeatother
% Scale images if necessary, so that they will not overflow the page
% margins by default, and it is still possible to overwrite the defaults
% using explicit options in \includegraphics[width, height, ...]{}
\setkeys{Gin}{width=\maxwidth,height=\maxheight,keepaspectratio}
\setlength{\parindent}{0pt}
\setlength{\parskip}{6pt plus 2pt minus 1pt}
\setlength{\emergencystretch}{3em}  % prevent overfull lines
\providecommand{\tightlist}{%
  \setlength{\itemsep}{0pt}\setlength{\parskip}{0pt}}
\setcounter{secnumdepth}{0}

\date{}

% Redefines (sub)paragraphs to behave more like sections
\ifx\paragraph\undefined\else
\let\oldparagraph\paragraph
\renewcommand{\paragraph}[1]{\oldparagraph{#1}\mbox{}}
\fi
\ifx\subparagraph\undefined\else
\let\oldsubparagraph\subparagraph
\renewcommand{\subparagraph}[1]{\oldsubparagraph{#1}\mbox{}}
\fi

\setlength{\oddsidemargin}{-0.1in}
\setlength{\topmargin}{-0.52truein} 
\setlength{\textheight}{9.15in} 
\setlength{\textwidth}{6.7in}

\usepackage[T1]{fontenc}
\usepackage{fourier}
\usepackage[sc]{mathpazo}
\linespread{1.05}         % Palatino needs more leading (space between lines)
\newcommand\Alfven{Alfv\'en }
\newcommand{\V}[1]{\mathbf{#1}} 


\usepackage{wrapfig}
\usepackage[square,numbers,sort&compress]{natbib}
\renewcommand{\cite}{\citep}
%\usepackage[psamsfonts]{amssymb}
%\usepackage{palatino}
%\usepackage{mathpazo}
%\usepackage{plasmadefs}

\hyphenation{wave-packet wave-packets}



\begin{document}

\section*{Budget Justification}

\subsection*{Personnel and Benefits}

Funds are requested to provide 1 month of summer salary per year for
the PI, Prof. Troy Carter and faculty Co-PIs Prof. Walter Gekelman and
Prof. George Morales.  Co-PI Dr. Steve Vincena is supported for 11
months at 67\% time (1 month of 100\% support is requested as part of
the NSF budget).  Other research staff are supported for 12 months at
67\% time: Dr. Bart Van Compernolle and Dr. Shreekrishna Tripathi.  As
a large fraction of the time of the research staff and technical staff
is spent fabricating equipment for BaPSF, the remaining percentage of
Dr. Vincena (for 11 months), Dr. Van Compernolle, and Dr. Tripathi's
time is covered under Fabrication (below).  The facility requires a
full-time administrator to handle the work load dealing with running
the facility and external users, as well as travel and workshops; this
is outside of the normal services provided by the University as part
of the F\&A charge.  Support for 100\% of Meg Murphy's salary is
therefore requested.  Support for a full-time IT specialist is
requested; this person would be in charge of hardware and software
development and maintainence, including Labview-based control software
for the facility.  Salary for other personnel are catagorized under
Fabrication, below.

 As the facility supports some of the
research of the local group, five full-time graduate students are
included in the budget.  The specific salaries and benefits for each
individual are determined by strict guidelines set by the University
of California and subject to rigorous review. Each listed salary is
based on the individual’s current level and increased each year for
cost of living and merit increase (5\% COLA for academic personnel,
4\% COLA for staff, 7\% for faculty merit increases which occur every
few years). Benefits are the actual benefits of each employee.  Fringe
benefits are charged to all senior and other new personnel at a
composite rate set by the Regents of University of California.  PI and
Co-I salaries are charged at a rate of 12.7\%. Postdoc salary is
charged at a rate of 19\%. Meanwhile, the graduate student researchers
are charged at 1.3\% during the academic months and 3.0\% during the
summer months.  Existing employee use actual rates.  Benefit rates are
increased 2\% each fiscal year.




\subsection*{Equipment and Fabrication}

Each year there are funds allocated for continuing modification to
current equipment. The actual needs in this category vary from year to
year (e.g. repairing/replacing pump-down stations, mechanical and
electrical components of the LAPD, etc.).  Each year, equipment in
support of user experiments and campaigns is designed and constructed
by the BaPSF staff. 

Other planned fabrication items are called out in
the budget and include: High frequency RF amplifiers for wave
excitation (e.g., whistler, lower hybrid, Alfv\'{e}n); Fabrication of
200 GHz interferometers (over the 5 year period funds to construct 5
new interferometers are requested to provide 5 axial locations of
line-integrated density measurements; a shielded fast-wave antenna to
be used for high-power fast wave launch (in support of local group and
campaign research); 3 valve/port boxes for insertion of large-sized
apparatus into LAPD; amplifiers for diagnostic signal conditioning
(low frequency and high frequency, called out separately); transisitor
switches for power supplies/drivers for wave excitation and biasing;
new probe drives that will allow three-axis positioning of probes in
LAPD; fabrication of probes (Langmuir, Mach, emissive, magnetic,
dipole, etc.) happens continuously, and funds are requested to support
this; vacuum hardware (KF clamps, o-rings, pumpdown ports, etc).  

In the first year funds are
requested to fabricate a new large area LaB$_6$ cathode source,
including internal (to the vacuum chamber) magnets.  The materials for
fabricating the source will be purchased in the first year with
construction continuing into the second year.  As part of the
fabrication budget, funds to support the the salary (all or part) of
several staff members is requested; the percentage time they spend on
fabrication of equipment for the facility is charged to a different
account on which no overhead is drawn (consistent with the rules on
fabrication of equipment at UCLA).  Under fabrication, funds to
support Dr. Vincena (33\% for 11 months), Dr. Tripathi (33\% for 12
months) and Dr. Van Compernolle (33\% for 12 months) are requested.
In addition, the Technical Director, Zoltan Lucky is supported 75\%
time under fabrication, the Project Scientist, Dr. Pat Pribyl, is
supported 100\%, as are the two technicians, Marvin Drandell (100\%)
and Tai Ly (100\%). 


  There is a budget item for replacement/upgrade
of basic facility equipment (e.g., scopes, meters). This is necessary,
as there comes a time when it is not cost effective to repair old
equipment and it must be replaced. 
Funds are requested for storage of data generated by
BaPSF users and the local group.  Experimental runs can generate up to 1 TB
per week of data; continuous investment is needed to provide
sufficient high-quality, backed-up storage space for data.  In
addition, funds for data analysis servers are requested.  The servers
are used both by the local group and by BaPSF users.  Funds are
requested for a new fiber-coupled spectrometer (and CCD) for Doppler
spectroscopy of ions (helium, argon) for flow and ion temperature
determination.  Funds for replacement and new vacuum pumps are
requested.  


\subsection*{Travel}

The travel budget allows the PIs, research staff, and graduate students
to attend major national scientific meetings such as APS-DPP and to
participate in international meetings such as the International
Conference on Plasma Physics (ICPP), European Physical Society Plasma
Physics meeting (EPS), the Interrelationship of Plasma
Experiments in the Laboratory and Space (IPELS), the Trieste Summer
School on Plasma Physics and major topical workshops. Travel to these
various meetings is necessary to present the work done at BaPSF to
wide audiences and to fulfill the “Broader Impact” goals of the
proposal.  Funds are requested for 7 domestic trips (APS DPP or AGU) and 3
international trips (e.g. IPELS or EPS Plasma Physics) per calendar
year. We have assumed that the trips will be 1 week long each in budgeting.   The trips will be made by the PI and Co-PIs, research staff and graduate
students who will report on the latest research taking place at BaPSF.


\subsection*{Other Direct Costs}

The materials and supplies category includes funds for necessary small
replacement parts as well as for electrical supplies; the amounts
requested are consistent with spending in these categories over the
last 5 year period of support for BaPSF. \$30k has been
allocated for scientific publications as a result of this project. The
additional direct costs include the following.  There are three laser
systems (two Nd:YAG lasers and a pulsed tunable dye laser) used in
support of several experiments and for diagnostics. All the lasers
must be kept under service contract, as a single repair on any of them
equals the price of a year-long service contract. The lasers are
fragile and usually break more than once a year. The service contract
also includes a yearly alignment to bring the laser up to its quoted
specifications.  Funds are requested for yearly replacement of laptops and computers, both for research staff use
as well as for use in the lab (connecting to RGA's, monitoring of
laboratory equipment and control systems).  Programmable power
supplies are used throughout the lab (e.g., for antenna RF drivers,
biasing circuits, Langmuir probe measurements); request for purchasing
(or replacing) 1 power supply per year is included.


Participant support costs are requested in order to
facilitate Campaign workshops and Users Group meetings at UCLA. One
event per year (either a campaign workshop or the biannual on-site
Users Group meeting) is envisioned.  The funds allocated will support
the travel of 10 people per year to these events (and may be used to
support partial payment of travel for more).


The GSR fee remissions for the current academic year (2015-2016) are
at \$15,440.48. An estimated 5\% increase per year is applied. We
request funds for GSR fee remission for 5 graduate students for three
academic quarters on each year of this project. \\[0.1truein]
{\ttfamily https://grad.ucla.edu/gss/library/1516remissionsgsr.pdf} \\
{\ttfamily https://grad.ucla.edu/gss/appm/gsr10stepscale.pdf} \\

The Technology Infrastructure Fee (TIF) is a consistently-applied direct charge that
is assessed to each and every campus activity unit, regardless of
funding source, including units identified as individual grant and
contract awards.  The TIF pays for campus communication services on
the basis of a monthly accounting of actual usage data.  These costs
are charged as direct costs and are not recovered as indirect costs.
TIF is based on a full time employee (FTE) and is calculated at \$33.28
per FTE per month.  

\subsection*{Indirect Costs}

Facilities and Administration (F\&A) costs are federally negotiated
rate applied to projects as a percentage of the direct costs.  The
UCLA F\&A rate on the Modified Total Direct Cost (MTDC) is 54\%.  MTDC
consists of all salaries and wages, fringe benefits, materials and
supplies, services, travel and the first \$25,000 of each subaward
regardless of the period covered by the sub-grant or subcontract.
Graduate student fee remissions, equipment over \$5,000, and
participant support costs are excluded from MTDC.  On April 27, 2011,
the University of California and the United States Department of
Health and Human Services (the responsible Federal audit agency)
entered into a new facilities and administrative (F\&A) cost rate
agreement for UCLA. This agreement establishes facilities and
administrative cost rates for the period July 1, 2010, through June
30, 2016.  The F\&A rate pertaining to this budget is 54\% of Modified
Total Direct Costs for on-campus research.  A copy of the F\&A rate
agreement can be accessed at: {\ttfamily
  http://www.research.ucla.edu/ocga/Documents/F\_A\_Rate\_Agreement\_4-27-11.pdf}.

\end{document}


%% \subsubsection*{UCLA Contribution}

%% The BaPSF occupies nearly the entire space of the UCLA Science and
%% Technology Research Building

%%  Aside from the usage of
%% nearly an entire building (the Science and Technology Research
%% Building) on Campus the Dean of the College of Letters and Science has
%% promised one quarter of teaching relief each year for the director of
%% the BaPSF.
